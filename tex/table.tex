\documentclass{article}
\begin{document}

\begin{tabular}{|r|l|}
  \hline
  7c0 & hexadecimal \\
  3700 & octal \\ \cline{2-2}
  11111000000 & binary \\
  \hline \hline
  1984 & decimal \\
  \hline
\end{tabular}

\begin{tabular}{|p{4.7cm}|}
  \hline
  Welcome to Boxy’s paragraph.
  We sincerely hope you’ll
  all enjoy the show.\\
  \hline
\end{tabular}

\begin{tabular}{@{} l @{}}
  \hline
  no leading space\\
  \hline
\end{tabular}

\begin{tabular}{l}
  \hline
  leading space left and right\\
  \hline
\end{tabular}
\begin{center}
\begin{tabular}{c r @{.} l}
  Pi expression &
  \multicolumn{2}{c}{Value} \\
  \hline
  $\pi$ & 3&1416 \\
  $\pi^{\pi}$ & 36&46 \\
  $(\pi^{\pi})^{\pi}$ & 80662&7 \\
\end{tabular}
\end{center}

\begin{tabular}{|c|c|}
  \hline
  \multicolumn{2}{|c|}{Ene} \\
  \hline
  Mene & Muh! \\
  \hline
\end{tabular}

Figure \ref{white} is na example of Pop-Art
\begin{figure}[!hbp]
  \makebox[\textwidth]{\framebox[5cm]{\rule{0pt}{5cm}}}
  \caption{Five by Five in Centimetres. \label{white}}
\end{figure}

Figure \ref{white} is na example of Pop-Art
\begin{figure}[!hbp]
  \makebox[\textwidth]{\framebox[5cm]{\rule{0pt}{5cm}}}
  \caption{Five by Five in Centimetres. \label{white}}
\end{figure}

\end{document}
