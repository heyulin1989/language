\documentclass{article}

\usepackage{amsmath}
\usepackage{amssymb}

\begin{document}
Add $a$ squared and $b$ squeared
to get $c$ squared. or, using a
more mathematical approach:
$\left(c^{2}=a^{2}+b^{2^{3^{4^{5^{6^{7^{8^{9^{10^{11^a{^b{^c{^d}}}}}}}}}}}}}\right)$


\TeX{} is pronounced as
\(\tau\epsilon\chi\). \\[6pt]
100m$^{3}$ of water\\[6pt]
this comes from my
\begin{math}
  \heartsuit
\end{math}


  Add $a$ squared and $b$ squared
  to get $c$ squared. Or, using
  a more mathematical approach:
  \begin{displaymath}
    c^{2}=a^{2}+b^{2}
  \end{displaymath}
  or you can type less with:
  \[a+b=c\]


  \begin{equation} \label{eq:eps}
    \epsilon > 0
  \end{equation}
  From (\ref{eq:eps}), we gather\ldots{}
  From \eqref{eq:eps} we
  do the same.

  \begin{flushright}
  $\lim_{n \to \infty}
  \sum_{k=1}^n \frac{1}{k^2}
  = \frac{\pi^2}{6}$
  \end{flushright}

  \begin{displaymath} \label{eq:eps}
    \lim_{n \to \infty}
    \sum_{k=1}^n \frac{1}{k^2}
    = \frac{\pi^2}{6}
  \end{displaymath}

  \begin{equation}
    \forall x \in \mathbf{R}:
    \qquad x^{2} \geq 0
  \end{equation}

  \begin{displaymath}
    x^2 \geq 0\qquad
    \textrm{for all }x\in\mathbb{R}
  \end{displaymath}
  \begin{center}
  $\lambda,\xi,\pi,\mu,\phi,\omega$
  
  $\Lambda,\Xi,\Pi,\qquad\Phi,\Omega$

  $a_{1}$ \qquad $x^{2}$ \qquad
  $e^{-\alpha t}$ \qquad
  $a^{3}_{ij}$\\
  $e^{x^2}\neq {e^x}^2$

  $\sqrt{x}$ \qquad
  $\sqrt{x^{2}+\sqrt{y}}$\qquad
  $\sqrt[3]{2}$ \\[3pt]
  $\surd[x^2 + y^2] + \sqrt{x^2+y^2}$\\
  $\overline{m+n}$\\
  $\underbrace{a+b+\cdots+z}_{26}$

  \end{center}
  \begin{center}
  ------------------------------\\
  \begin{displaymath}
    \binom{n}{k}\qquad\mathrm{C}_n^k
  \end{displaymath}
  \\
  ------------------------------
  \end{center}
  \begin{center}
  ------------------------------\\
  \begin{displaymath}
    \int f_N(x) \stackrel{!}{=} 1
  \end{displaymath}
  \\
  ------------------------------
  \end{center}
  \begin{center}
  ------------------------------\\
  \begin{displaymath}
    \sum_{i=1}^{n} \qquad
    \int_{0}^{\frac{\pi}{2}} \qquad
    \prod_\epsilon
  \end{displaymath}
  \\
  ------------------------------
  \end{center}

  \begin{center}
  ------------------------------\\
  \begin{displaymath}
    \sum_{\substack{0<i< n \\ 1<j<m}}
    P(i,j)=
    \sum_{\begin{subarray}{1}
        i \in I\\
        1<j<m
      \end{subarray}}
      Q(i,j)
  \end{displaymath}
  \\
  ------------------------------
  \end{center}
  \begin{center}
  \begin{displaymath}
    1 + \left( \frac{1}{1-x^2}
    \right)^3
  \end{displaymath}
  \\
  ------------------------------
  \end{center}

  \begin{center}
    $\Big( (x+1) (x-1) \Big)^2$\\
    $\big(\Big(\bigg(\Bigg($\\
    $\big\}\Big\}\bigg\}\Bigg\}$\\
    $\big\|\Big\|\bigg\|\Bigg\|$\\
    $\big/\Big/\bigg/\Bigg/$\\
    ------------------------------
  \end{center}
  \begin{center}
    \begin{displaymath}
      x_{1},\ldots,x_{n}\qquad
      x_{1}+\ldots+x_{n}
    \end{displaymath}
    -------------------------------------------------------------------
  \end{center}
  \begin{center}
    \newcommand{\ud}{mathrm{d}}
    \begin{displaymath}
      \int\!\!\!\int_{D} g(x,y) \,\ud x\,\ud y
    \end{displaymath}
    -------------------------------------------------------------------
    \begin{displaymath}
      \iint_{D} g(x,y) \,\ud x \,\ud y
    \end{displaymath}
    -------------------------------------------------------------------
    \begin{displaymath}
      \mathbf{X} =
      \left( \begin{array}{ccc}
        x_{11} & x_{12} & \ldots \\
        x_{21} & x_{22} & \ldots \\
        \vdots & \vdots & \ddots
      \end{array} \right)
    \end{displaymath}
    -------------------------------------------------------------------
    \begin{displaymath}
      y = \left\{ \begin{array}{ll}
        a & \textrm{if $d>c$}\\
        b+x & \textrm{in the morning}\\
        l & \textrm{all day long}
      \end{array} \right\}
      \end{displaymath}

    \begin{displaymath}
      \left(\begin{array}{c|c}
        1 & 2 \\
        \hline
	    3 & 4
      \end{array}\right)
    \end{displaymath}
  \end{center}
\end{document}
